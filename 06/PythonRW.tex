\documentclass[11pt]{article}
\usepackage[margin=1in]{geometry}
\usepackage{graphics, graphicx, cite, fancybox, setspace}
\usepackage{amsfonts, amssymb, amsmath, latexsym, epic, eepic, url,enumitem}
\usepackage{wrapfig, subfigure}
\usepackage{multirow}
\usepackage{palatino}
\usepackage[mathscr]{euscript}
\usepackage{amsfonts}
\usepackage[bf]{caption}
\usepackage{dcolumn}
\usepackage[numbers]{natbib}
\usepackage[usenames,dvipsnames]{color}
%\usepackage[normalem]{ulem}
%\usepackage{enumitem}
\usepackage{acronym}
%\usepackage[centertags]{amsmath}
%\usepackage{amsfonts,amssymb}
\usepackage{accents}
\usepackage{tikz}
\usetikzlibrary{shapes,arrows}
% Default fixed font does not support bold face
\DeclareFixedFont{\ttb}{T1}{txtt}{bx}{n}{12} % for bold
\DeclareFixedFont{\ttm}{T1}{txtt}{m}{n}{12}  % for normal

% Custom colors
\usepackage{color}
\definecolor{deepblue}{rgb}{0,0,0.5}
\definecolor{deepred}{rgb}{0.6,0,0}
\definecolor{deepgreen}{rgb}{0,0.5,0}

\usepackage{listings}

% Python style for highlighting
\newcommand\pythonstyle{\lstset{
language=Python,
basicstyle=\ttm,
otherkeywords={self},             % Add keywords here
keywordstyle=\ttb\color{deepblue},
emph={MyClass,__init__},          % Custom highlighting
emphstyle=\ttb\color{deepred},    % Custom highlighting style
stringstyle=\color{deepgreen},
frame=tb,                         % Any extra options here
showstringspaces=false            % 
}}


% Python environment
\lstnewenvironment{python}[1][]
{
\pythonstyle
\lstset{#1}
}
{}

% Python for external files
\newcommand\pythonexternal[2][]{{
\pythonstyle
\lstinputlisting[#1]{#2}}}

% Python for inline
\newcommand\pythoninline[1]{{\pythonstyle\lstinline!#1!}}
%\renewcommand{\baselinestretch}{1.1}
%
\newcommand{\molar}[1]{\underaccent{\bar}{#1}}
\def\Vm{\molar{V}}
%\renewcommand{\figurename}{Algorithm}
\newcommand{\cfa}[1]{\color{red}{#1}\color{black}}
\newcommand{\eve}[1]{\color{blue}{#1}\color{black}}
\newcommand{\uu}[1]{{\boldsymbol #1}}
\newcommand{\tpd}[3]
{\left(\frac{\partial {#1}}{\partial {#2}}\right)_{#3}}
\def\ttau{\uu{\tau}}
\def\bb{\uu{b}}
\def\nb{\uu{n}}
\def\pb{\uu{p}}
\def\wb{\uu{w}}
\def\xb{\uu{x}}
\def\yb{\uu{y}}
\def\vb{\uu{v}}
\def\zb{\uu{z}}
\def\Xb{\uu{X}}
\def\etab{\uu{\eta}}
\def\thetab{\uu{\theta}}
\def\lambdab{\uu{\lambda}}
\def\gammab{\uu{\gamma}}
\def\taub{\uu{\tau}}
\def\phib{\uu{\phi}}
\def\psib{\uu{\psi}}
\def\nub{\uu{\nu}}
\def\nablab{\uu{\nabla}}
\def\varphib{\uu{\varphi}}
\def\Ab{\uu{A}}
\def\Bb{\uu{B}}
\def\Gb{\uu{G}}
\def\Vb{\uu{V}}
\def\Fb{\uu{F}}
\def\Jb{\uu{J}}
\def\Rb{\uu{R}}
\def\Tb{\uu{T}}
\def\Mb{\uu{M}}
\def\rb{\uu{r}}
\tikzstyle{decision} = [diamond, draw, fill=blue!20, 
    text width=4.5em, text badly centered, node distance=3cm, inner sep=0pt]
\tikzstyle{block} = [rectangle, draw, fill=blue!20, 
    text width=7em, text centered, rounded corners, minimum height=4em]
\tikzstyle{gblock} = [rectangle, draw, fill=green!20,
    text width=7em, text centered, rounded corners, minimum height=4em]
\tikzstyle{line} = [draw, -latex']
\tikzstyle{cloud} = [draw, ellipse,fill=red!20, node distance=3cm,
    minimum height=2em]
\hyphenation{mol-e-cules}
\newenvironment{packed_item}{
\begin{itemize}
  \setlength{\itemsep}{0pt}
  \setlength{\parskip}{0pt}
  \setlength{\parsep}{0pt}
}{\end{itemize}}

\newenvironment{packed_enum}{
\begin{enumerate}
  \setlength{\itemsep}{0pt}
  \setlength{\parskip}{0pt}
  \setlength{\parsep}{0pt}
}{\end{enumerate}}

\renewcommand\thesection{\Alph{section}}

\newif \ifshowsolutions
\showsolutionstrue
\showsolutionsfalse

\begin{document}

\begin{centering}
{\sc D\ R\ E\ X\ E\ L\ \ \ U\ N\ I\ V\ E\ R\ S\ I\ T\ Y}\\
College of Engineering\\
ENGR 131 -- Programming \\
Winter 18-19\\
A Suggested Programming Exercise:  Visualizing and Characterizing Random Walks\\
Cameron F. Abrams -- cfa22@drexel.edu\\
Department of Chemical and Biological Engineering\\
\ifshowsolutions
\textcolor{blue}{S\ O\ L\ U\ T\ I\ O\ N\ S}\\
\fi
\end{centering}

\section*{Summary}
In this assignment, you'll develop a Python program to generate and analyze random walks in 3-D space.  The main behavior we will analyze is the relationship between the {\bf end-to-end distance}, termed $d$, and the {\bf number of steps} $n$ of a random walk.  
The major Python programming skills we will learn about in this exercise are {\em multidimensional arrays} and random-number generation using {\tt NumPy}, as well as {\em vector operations} in 3-D space and 3-D plotting.

\section*{Background}

Random walks are common mathematical models used to describe phenomena in physics and engineering.  The ceaseless thermal agitation of molecules in liquids means that each molecule executes a 3D random walk, and this type of motion is the physical picture underlying the phenomenon of diffusion.  

In 3D space, one can imagine a random walk as a sequence of positions: $\rb_0,\ \rb_1,\ \dots\ rb_N$, where the distance between adjacent positions $\rb_{i}$ and $\rb_{i+1}$ is small.  We will take as our unit of measurement a constant step length in the random walk.  This means that the {\em norm} of the vector from any position in the walk $\rb_i$ to its next position $\rb_{i+1}$ is 1.  To work in 3D, it is best to work in {\em polar} coordinates $(r,\theta,\phi)$.  The standard mapping from Cartesian $(x,y,z)$ and polar coordinates is
%
\begin{align*}
x & = r \sin\theta \cos\phi\\
y & = r \sin\theta \sin\phi\\
z & = r \cos\theta
\end{align*}
%
A step in a random walk occurs in a random direction in 3D space.  However, we can't choose a random value for this direction uniformly in Cartesian space by uniformly sampling displacements along $x$, $y$, and $z$.   Instead, given position $\rb_i$ and step length $R$ (here this is 1), we first choose a uniform random number $-\pi<\phi<\pi$ and another $-1<\cos(\theta)<1$, and use polar coordinates to generate the displacements along $x$, $y$, and $z$ these two numbers dictate.  

\section*{The Assignment}

\begin{enumerate}
	\item xxx
\end{enumerate}

\end{document}
